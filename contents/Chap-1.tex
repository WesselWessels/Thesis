%!TEX root = ../USthesis_Masters.tex
\chapter{Introduction}
\label{chp:Intro}

Bitcoin \cite{Nakamoto2008} is a peer-to-peer decentralised digital payment mechanism that was introduced in a paper published by a person or group called Satoshi Nakamoto in 2008. Bitcoin aims to be an alternative to traditional centralised online payment mechanisms like credit cards and PayPal \cite{PayPal2015}. 

We focus on testing the viability of Bitcoin as payment mechanism on social media platforms, specifically on a mobile platfom. 


%%%%%%%%%%%%%%%%%%%%%%%%%%%%%%%%%%%%%%%%%%%%%%%%%%%%%%%%%%%%%%%%%%%%%%%
\section{Background}
	

\section{Related Work}

\section{Objectives}
% In granular or particle flow simulations with Discrete Element Method (DEM),
% the mechanical behavior of a system of particles are simulated. The basic
% building blocks of DEM are finite sized particles and walls. It is generally
% classified into two basically different approaches.

% The first is the ``hard sphere'', event-driven method
% \citep[e.g.][]{Luding-1994, Luding-2004}, where particles are assumed to be
% perfectly rigid and they follow an undisturbed motion until a collision
% occurs. Due to the rigidity of the interaction, the collisions occur
% instantaneously with accompanying momentum transfer. It is mainly used for
% collisional, dissipative granular gases.

% The second is the so-called ``soft particle'' molecular dynamics pioneered by
% \citet{Cundall-1979}, where the particles are allowed to overlap or penetrate
% each other. Constrains on the physical space that a particle can occupy at a
% specific time is included with contact or penalty forces related to the
% amount of overlap and contact velocity between particles or between particles
% and walls. The motion of the system is modelled by the integration of
% Newton-Euler equations for motion of every individual particle.
