%!TEX root = ../USthesis_Masters.tex

\chapter{Conclusion}
\label{chp:Conclusion}


%%%%%%%%%%%%%%%%%%%%%%%%%%%%%%%%%%%%%%%%%%%%%%%%%%%%%%%%%%%%%%%%%%%%%%%
In the area of small mobile payments on a social media platform, Bitcoin was found to be a viable payment mechanism. Bitcoin was shown to be able to process a transaction with a fee of about R0.01 on a R1 transaction, where the same transaction with an SMS payment would cost at least R0.90. 

With the payment framework that was implemented on WeChat, we were successfully able to manage Bitcoin payments as a service. The Bitcoin walllet application that was built to run on WeChat was able to function as a basic Bitcoin wallet application. It was also able to make payments directly to the use-case gamebooks application.

Despite the constrained environment of the chat application, the user feedback for the overall experience of using Bitcoin as a payment mechanism was positive. Further constraints, like relying on third-party services that increase response time increased the negative feedback regarding the user experience. 

\section{Contributions}




\section{Future Work}

The implementation of the payment framework in this project is intended to be a proof of concept. As such, there was not enough focus on security, especially in terms of the storage of Bitcoin private keys. If a framework like this is implemented on a production platform, extreme attention should be given to security and possible exploitations.

Furthermore, a platform like the one built would be better if it relied less on multiple third-party platforms for services like Bitcoin blockchain access and QR-code decoding. For the sake of security and control, services like this would be better to be implemented by the organisation that builds the platform. Having control of a bigger part of the system reduces the points of failure.


