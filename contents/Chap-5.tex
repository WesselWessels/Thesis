%!TEX root = ../USthesis_Masters.tex

\chapter{Tests}
\label{chp:Tests}


%%%%%%%%%%%%%%%%%%%%%%%%%%%%%%%%%%%%%%%%%%%%%%%%%%%%%%%%%%%%%%%%%%%%%%%
\section{Quantitative}

This project is an implementation of a platform on top of an existing social media platform WeChat. We do not have access to the back-end systems of WeChat. This makes it hard to do many of the quantitative tests required for the application itself. They will require qualitative tests. The REST API however, can be tested thoroughly using unit tests. 

\subsection{REST API unit tests}

There are tools and frameworks available for testing REST API's. Just like when making the REST API, we decided to rather write the unit tests ourselves without using a framework in order to get a better understanding of how unit tests work.

Thus, we wrote the entire REST API unit tests in a PHP script that calls all our REST endpoints with inputs that we know what the outputs should be. Each test should also confirm that the correct HTTP status code, as shown in table \ref{tbl:http_status_codes},  is returned. In total, there are 39 different tests that excecute. 

\begin{table}
	\begin{center}
		\begin{tabular}	{ | c | c | p{5cm}|}
		\hline
		Status Code & Text & Description \\ \hline
		200 & OK & Request succeeded \\ \hline
		400 & Bad Request & Request not understood due to malformed syntax \\ \hline
		401 & Unauthorized & The request requires user authentication \\ \hline
		404 & Not Found & Request does not exist \\ \hline
		
	   % \label{fig:htaccess_file}
		\end{tabular}
		\caption{Table of selected HTTP status codes} 
		\label{tbl:http_status_codes}
	\end{center}
\end{table}



\subsubsection{Testing Authentication}
In order to test the authentication of the developer with our REST API, we wrote a simple endpoint called /test that returns an HTTP status code ``200 OK'' if the authentication was successful and the corresponding status code if it is unsuccessful.



The following are the tests that should not authenticate:

\begin{itemize}
	\item Duplicate Request
	\item Missing api\_key
	\item Invalid api\_key
	\item Missing nonce
	\item Missing timestamp
	\item Missing signature
	\item Invalid signature
\end{itemize}

Since all of these tests fail to authenticate, they should all return an HTTP status code ``401 Unauthorized''.

All further tests assume that the authentication happens successfully.

\subsubsection{Testing /payment}

To test the /payment endpoint, we make a payment request with an amount and a description. For the result, we expect a valid Bitcoin address with the same amount and description that we provided. We use a function in bitcoin-lib-php library to validate the Bitcoin address. 

The following are tests that should fail:

\begin{itemize}
	\item Wrong Method (GET instead of POST)
	\item amount\_sat smaller or equal to 0
	\item amount\_sat not an integer
	\item Missing description
	\item Description too long
\end{itemize}

These tests should all return an HTTP status code ``400 Bad Request''.

\subsubsection{Testing /payment/\{TRANSACTION\_ID\}}

To test this method, we need to call it with a valid transaction ID that corresponds to the api\_key that created the transaction. If the ID does not correspond with the api\_key, the test should return an HTTP status code ``401 Unauthorized''.

\subsubsection{Testing /payment/\{PUBLIC\_ADDRESS\}}

Testing this method is very similiar to the /payment/\{TRANSACTION\_ID\}. The public address must correspond with the api\_key that created the transaction. 

The following tests should fail:

\begin{itemize}
	\item Wrong Public Address (401 Unauthorized)
	\item Invalid Public Address (400 Bad Request)
\end{itemize}

\subsubsection{Testing /balance}

This method only has no arguments. Thus, if the authentication succeeds, it should return an HTTP status code ``200 OK'' and the balance.

\subsubsection{Testing /payout}

This method will always return an HTTP status code ``200 OK'' with the result in the JSON response. 

The following tests should fail:

\begin{itemize}
	\item No Address Given
	\item Invalid Address
\end{itemize}

\subsubsection{Testing Invalid API Method}

All these tests should return an HTTP status code ``404 Not Found'':

\begin{itemize}
	\item No Method Given
	\item Testing ``/''
	\item Testing /asdfasdf
	\item Testing /asdfasdf/asdfasdf
	\item Testing /balance/asdfasdf
	\item Testing /payment/asdfasdf
\end{itemize}	

\subsection{Timing Tests}

To test the loading times, we sent messages to the Gamebooks and Wallet Application and measured how long it took to get a response. We conducted three loading tests:

\begin{enumerate}
	\item{Gamebooks Menu - minimal database queries}
	\item{Browse Gamebooks - substantial database queries}
	\item{Wallet Balance - substantial database and Bitcoin queries}
\end{enumerate}

In conjuction with testing our Official Accounts, we also tested a popular Official Account from the sport broadcast company SuperSport to have a baseline for comparison.

\begin{table}
	\begin{center}
		\begin{tabular}	{ | c| c | c | c| c|}
		\hline
		 &Gamebooks Menu & Gamebooks Browse & Wallet Balance & SuperSport\\ \hline
		 	& 8.8 & 6.8 & 12.4 & 5.9 \\ \hline
			& 7 & 6.4 & 8.5 & 6.3 \\ \hline
			& 9.1 & 6.1 & 14.3 & 3.9 \\ \hline
			& 6.1 & 6.6 & 7.1 & 5.8 \\ \hline
			& 10.7 & 8.5 & 10.5 & 4.9 \\ \hline
			& 4.7 & 8.3 & 7.6 & 6.1 \\ \hline
			& 4.2 & 7.8 & 8 & 8.4 \\ \hline
			& 6.4 & 7 & 7.3 & 5.2 \\ \hline
			& 11.3 & 6.3 & 8.3 & 8.4 \\ \hline
			& 6.6 & 6.1 & 8.1 & 7.9 \\ \hline 
			\textbf{Average} & \textbf{7.49} & \textbf{6.99} &  \textbf{9.21} & \textbf{6.29}\\ \hline
			\textbf{Std. Dev.} & \textbf{2.40} & \textbf{0.90} & \textbf{2.41} & \textbf{1.52} \\ \hline
		
		
	   % \label{fig:htaccess_file}
		\end{tabular}
		\caption{Table of loading times in seconds} 
		\label{tbl:loading_times}
	\end{center}
\end{table}

As can be seen from table \ref{tbl:loading_times}, the loading times for our Official Accounts are very high. Getting the wallet balance took the longest of all the requests that we tested. The reason for the long time is all the steps that have to be completed for this request. First, the request is sent from the WeChat Application on the mobile phone. That request has to go to WeChat's server in China. If we ping wechat.com, we get an average round-trip time of 492ms. WeChat then has to validate the request and send it to our server in Singapore. 

Our server then has to validate the request from WeChat and do database operations to get the address of the user. Since we are using the third-part service Chain.com for Bitcoin queries, we have to make a GET request to their services. This request has to check the Bitcoin Blockchain to determine the balance of the spesified address. Checking the balance and the round trip of the request seems to drastically increase the time of the interaction, since the request that involved Bitcoin took roughly two seconds longer to complete. We used a webpage that we wrote to test the Chain.com API to test how long a ``balance'' call of an address takes, and found that it took an average of 2.5 seconds.

Fortunately, the timing tests of our baseline WeChat Official Account, SuperSport, produced very similar results to our tests. SuperSport performed 10\% faster than our average fastest test and 32\% faster than our average slowest test. Although the SuperSport Official Account performed faster than ours, it still performed very slow compared to a dedicated webpage loading time. The average loading time that we measured on webpage running on the same server as the backend of the Wallet and Gamebooks Apps was 2.7 seconds. This loading time is drastically shorter than the loading time of the requests in WeChat, and as such we can conclude that WeChat's Official Accounts are not the most ideal platform for speed. A web-app will outperform WeChat based on the tests that we have done. 

	
\section{Qualitative}

Since we are implementing a system that is used by people, a large part of our testing is qualitative.

We wrote a guide that goes step-by-step through the processes of using the systems that we have built.

The user is asked to add the Wallet Application inside WeChat and then load some Testnet Bitcoin to their wallet from a website that gives free Testnet away. This is to let them understand the concept of loading Bitcoin to their Wallet. They are then asked to write a small Gamebook on the Gamebook Author page. They only need to write three pages: one start page that links to two pages.

After they finished the Gamebook, they are asked to add the Gamebooks Application inside WeChat. They should then navigate in the Gamebooks Application to the small story they just wrote and buy it. When they select it, they will receive a payment ID that they must enter in the Wallet Application and confirm the payment. They should then be able to read the story they just wrote and bought.

To get feedback, we wrote a Google Form to make a questionnaire. The advantage of using a Google Form is that it directly outputs the answers of the users in a single spreadsheet.

The main things we wanted to learn from the feedback was how easy the users where able to use the payment mechanism and how viable it is compared to traditional payment mechanisms.

\subsection{Feedback}

90\% of users used an iPhone when testing. The reason for this being so high is that a phone was offered for them to use in case they didn't have WeChat installed on their own phone. 10\% of users had an Android device, and the test was successful.

60\% of users never used WeChat before this test, and the other 40\% have used it before, but they no longer use it.

20\% of users are familiar with Bitcoin, with 70\% saying they have heard about it and 10\% saying they have never heard about it. 

90\% of users where able to add the Wallet and Gamebooks Applications within WeChat on the first try, with 10\% being able to add it with some trouble.

60\% of users said that the Wallet Application is easier to use than traditional payment mechanisms, with 20\% saying they are the same an 10\% said that traditional payment mechanisms are easier.

50\% of users said that the Wallet Application is faster than traditional payment mechanisms, with 30\% of users saying they are the same and 20\% saying that traditional mechanisms are faster.

We asked users what payment mechanism they would prefer to use assuming that Bitcoin is more easily accessible. 50\% of users said they would prefer Paypal, with 40\% of users saying they prefer the Wallet Application and 10\% saying they prefer a debit card.

The users were asked to rate the Wallet Application overall out of 10. The average rating is 7.8, with the mode being 7 and a standard deviation of 1.48.

For the Gamebooks Application, we only look at a rough experience. All of the users were able to write a Gamebook on the author website, with the majority saying it was easy to write. All of the users were able to select the book they wrote on the app, purchase it and read it.

The users were also asked to rate the Gamebooks Application overall out of 10. The average rating is 8, with the mode also 8 and a standard deviation of 1.33.

For the Gamebooks Author page, the average rating is 8.4, with the mode being 9 and the standard deviation being 0.97.


\begin{table}
	\begin{center}
		\begin{tabular}	{ | c| c | c | c|}
		\hline
		 &Average & Mode & Std. Dev. \\ \hline
		 Wallet & 7.8 & 7 & 1.48 \\ \hline
		 Gamebooks App& 8 & 8 & 1.33\\ \hline
		 Author Page & 8.4 & 9 & 0.97\\ \hline

		
		
	   % \label{fig:htaccess_file}
		\end{tabular}
		\caption{Table of ratings of individual parts of the system} 
		\label{tbl:ratings}
	\end{center}
\end{table}


\subsection{Feedback Interpretation}

The response to the applications that was tested was generally positive. The majority of the users quickly grasped the concept of making a payment from the Wallet Application to a 3'rd party application. Most of the users were very optimistic about the concept of using Bitcoin as payment on the on a social platform. The majority also like the Gamebooks application in conjunction with the Author page. 

The most negative feedback came from the response that only 40\% of users said that they would prefer to use the Wallet Application over traditional payment mechanisms, even though 60\% said it is easier and 50\% said it is faster. From user's responses, it seems that many of the users do not trust Bitcoin as a payment mechanism. Many users also disliked the user interface of the WeChat platform and also complained that responses are very slow.




